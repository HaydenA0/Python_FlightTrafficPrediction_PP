
\chapter{General Introduction}
\label{chap:introduction}

The global aviation industry operates as a critical artery of the world economy, facilitating commerce, tourism, and connectivity. Its performance is a key barometer of economic health, yet it is also highly susceptible to global events. The COVID-19 pandemic represented an unprecedented disruption, grounding fleets and causing a historic contraction in air traffic. In the wake of this event, airport authorities and operators face the immense challenge of navigating a complex and volatile recovery. This new reality demands a paradigm shift from traditional, retrospective analysis to agile, data-driven, and forward-looking decision-making.

It is within this dynamic context that my end-of-year internship took place at the \textbf{Office National Des Aéroports (ONDA)}, the public establishment responsible for the management and development of Morocco's airport infrastructure. Faced with the limitations of manual data analysis and the critical need for predictive insights, ONDA presented a clear mandate: to leverage Artificial Intelligence to better understand and anticipate flight traffic dynamics. To address this challenge, I was tasked with the mission of designing, developing, and deploying an end-to-end \textbf{Flight Traffic Analysis and Prediction Dashboard}.

This project stands at the intersection of data engineering, machine learning, and software development. The primary goal was not merely to analyze historical data, but to operationalize predictive intelligence through a robust and accessible application. The core technical objectives defined for this mission were as follows:

The first objective was to develop a data processing and feature engineering pipeline capable of transforming raw time-series flight data into a structured dataset suitable for machine learning. The pipeline incorporated methods to represent cyclical temporal patterns and encode contextual factors such as the distinct phases of the COVID-19 pandemic.

\begin{figure}[h]
    \centering
    \includegraphics[width=0.9\textwidth]{../images/organigrame.png}
    \caption*{L'Organigrame d'ONDA}
\end{figure}



The second objective was to implement and evaluate several machine learning models for flight traffic forecasting. This included establishing a linear baseline using Ridge Regression and developing a non-linear model based on the LightGBM framework. Model performance was assessed through comparative analysis to determine predictive accuracy and robustness.

The final objective was to deploy the trained models in an interactive web application built with Dash Plotly. The application enabled on-demand forecasts and analytical exploration of model outputs, facilitating data-driven decision-making for end users.

This report will detail the complete methodology, from initial problem definition to final deployment, following the Cross-Industry Standard Process for Data Mining (CRISP-DM). The document is structured as follows:

\begin{itemize}
    \item \textbf{Chapter 2} Presents the general context of the project, including a detailed presentation of ONDA, the project's scope, and the work methodology adopted.
    
    \item \textbf{Chapter 3} Details the analysis of the existing system and the comprehensive specification of the functional and non-functional requirements for the new solution.
    
    \item \textbf{Chapter 4} Outlines the design phase, justifying the key technical choices and presenting the global system architecture.
    
    \item \textbf{Chapter 5} Describes the implementation of the solution, including the technology stack, the technical architecture, and the results obtained from the predictive models.
    
    \item Finally, a \textbf{General Conclusion} summarizes the achievements of the project, discusses the challenges encountered, and proposes perspectives for future work.
\end{itemize}
