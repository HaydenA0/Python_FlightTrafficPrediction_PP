% --- This goes into sections/06_conclusion.tex ---

\chapter{General Conclusion}
\label{chap:conclusion}

This internship project addressed a central analytical challenge at the Office National des Aéroports (ONDA): transitioning from a manual, retrospective mode of flight traffic analysis toward a predictive, data-driven framework capable of supporting strategic and operational decision-making. Guided by the CRISP-DM methodology, the work encompassed all stages of a modern data science pipeline—from data acquisition and preprocessing to model development, evaluation, and deployment within an interactive analytical platform.

\section{Summary of Achievements}

The primary objective was to construct an end-to-end predictive system capable of forecasting daily flight operations across airports in Morocco. This goal was met through the implementation of a robust, automated data pipeline and the development of machine learning models designed to capture the complex temporal and contextual dynamics of flight activity.

A key contribution of this project was the development of a feature engineering framework that accurately modeled both cyclical temporal effects (e.g., weekly and seasonal variations) and pandemic-related structural shifts. These features proved essential for improving the explanatory power of the models and ensuring generalization across different time periods.

Two distinct modeling approaches were implemented and systematically compared. The Ridge Regression model established a linear performance baseline, achieving an accuracy of approximately \textbf{78\%} with a mean computation time of under \textbf{100 milliseconds} per prediction. In contrast, the LightGBM model achieved substantially higher accuracy, approximately \textbf{92\%}, by effectively capturing non-linear dependencies and complex interactions within the data. This improvement came with a higher computational cost—an average inference time of roughly \textbf{4 seconds} per forecast—but remained within acceptable operational limits for interactive use. This trade-off between model complexity, predictive accuracy, and response time was carefully evaluated to balance interpretability and performance within the deployed system.

The final deliverable integrated these trained models into a fully interactive \texttt{Dash Plotly} web application. The platform enables users to visualize historical patterns, generate on-demand forecasts, and access key performance indicators (KPIs) without requiring technical expertise. This integration represents a major step toward operationalizing data science within ONDA’s decision-making processes.

\section{Challenges and Lessons Learned}

Throughout the project, several technical challenges were encountered and systematically addressed. Managing data heteroscedasticity required the application of a logarithmic transformation to stabilize variance across airports of different sizes. Iterative feature engineering was essential for improving model sensitivity to pandemic-related effects and long-term seasonal trends. Finally, ensuring a seamless user experience demanded thoughtful interface design to translate complex analytical results into clear, actionable insights for non-technical users. These challenges underscored the importance of iterative development and cross-disciplinary collaboration in data science projects of operational relevance.

\section{Future Work and Perspectives}

This project establishes a solid foundation for future analytics initiatives within ONDA. Several promising directions for extension have been identified:

\begin{itemize}
    \item \textbf{Automation through MLOps:} Implementing a continuous integration and deployment (CI/CD) pipeline to automate data ingestion, model retraining, and deployment as new flight data becomes available.
    
    \item \textbf{Integration of External Data Sources:} Incorporating exogenous variables such as economic indicators, tourism statistics, fuel prices, and event schedules to enhance model robustness and forecast precision.
    
    \item \textbf{Exploration of Advanced Architectures:} Investigating deep learning models such as Long Short-Term Memory (LSTM) networks or Transformer-based architectures, which may better capture long-range dependencies and multi-seasonal patterns in flight traffic data.
    
    \item \textbf{Extension of Dashboard Capabilities:} Enhancing the web application to support scenario-based simulations and anomaly detection, providing users with proactive insights into potential disruptions or deviations from expected operational patterns.
\end{itemize}

\section{Concluding Remarks}

In summary, this project achieved its core goal of developing a scalable, accurate, and interpretable forecasting system for national airport operations. The comparative analysis of linear and non-linear models demonstrated that while simple linear methods can deliver fast and reasonably accurate predictions, advanced ensemble techniques like LightGBM yield significantly higher precision at a modest computational cost. The successful deployment of the predictive system into an interactive, bilingual platform marks an important step toward institutionalizing data-driven decision-making within ONDA, setting the stage for continued innovation in predictive analytics and intelligent infrastructure management.

